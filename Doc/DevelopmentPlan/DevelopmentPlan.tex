\documentclass[12pt,fleqn]{article}

\usepackage{fullpage}
\usepackage[round]{natbib}
\usepackage{multirow}
\usepackage{booktabs}
\usepackage{tabularx}
\usepackage{tcolorbox}
\usepackage{graphicx}
\usepackage{float}
\usepackage{hyperref}
\usepackage{color, soul}
\usepackage{caption}
\usepackage[utf8]{inputenc}
\usepackage{array}
\usepackage{longtable}
\usepackage{indentfirst}

\newcolumntype{L}{>{\centering\arraybackslash}m{5cm}}
\newcolumntype{l}{>{\centering\arraybackslash}m{3cm}}

\hypersetup{
    colorlinks = true,
    citecolor=black,
    filecolor=black,
    linkcolor=blue,
    urlcolor=blue
}
\oddsidemargin 0mm
\evensidemargin 0mm
\textwidth 160mm
\textheight 200mm

\newcommand\submissionDate{February 4, 2022}

\usepackage{fancyhdr}
\fancyhead[L]{\submissionDate{}}
\fancyhead[C]{SE 3XA3: Software Project Management}
\fancyhead[R]{Lab L02: Group 15}
\setlength{\headheight}{52pt}
\renewcommand{\headrulewidth}{0.2pt}
\pagestyle{fancy}
\usepackage[margin=2.5cm,headsep=.2in ]{geometry}

\pagenumbering{arabic}
\newcounter{stepnum}


\title{SE 3XA3: Development Plan\\CryptoMetrics}

\author{Team 15
		\\ Saif Fadhel, fadhels
		\\ Vanshaj Verma, vermav2
		\\ Himanshu Aggarwal, aggarwah
}

\date{\today}


\begin{document}
\maketitle
\vspace*{\fill}
\begin{table}[hp]
\centering
\caption{Revision History} \label{TblRevisionHistory}
\begin{tabularx}{\textwidth}{llX}
\toprule
\textbf{Date} & \textbf{Developer(s)} & \textbf{Change}\\
\midrule
Feb. 04 & Saif, Himanshu, Vansh & Create and reviewed document\\
\midrule
Apr. 06 & Saif & Changed title to include project name\\
\midrule
Apr. 07 & Saif, Himanshu, Vansh & Added purpose of the document\\
\bottomrule
\end{tabularx}
\end{table}
\vspace*{\fill}


\newpage

\maketitle
\thispagestyle{empty}
\pagebreak

\noindent Today's virtual currencies, such as cryptocurrencies or other digital currencies, have significantly changed the way we view the concept of money. There is a growing interest in cryptocurrency worldwide, but without any knowledge of the projected trends, it is easy to make uneducated investments and lose hard earned money. Our proposed project, CryptoMetrics, will help users make more educated investments by providing a dashboard that displays historical and real-time cryptocurrency prices. \textcolor{red}{The purpose of this document is to outline the team meeting plan, communication plan, the roles of the team members, the workflow plan to publish git work, the proof of concept demonstration plan, and the technology used to develop the project.} 

\section{Team Meeting Plan}
\noindent The team will be meeting virtually two times a week for two hours during the 3XA3 lab times, either virtually using Discord or Microsoft Teams, or in-person in ITB 236. If more time is needed to discuss specific issues, Discord will be used to schedule additional meetings amongst the team members. The meeting roles for roles each team member are described as follows: \\

\begin{table}[h!]
    \centering
    \begin{tabular}{|c|c|}
    \hline
    \multicolumn{1}{|c|}{Name}              & \multicolumn{1}{c|}{Role}                \\ \hline
    \multicolumn{1}{|c|}{Himanshu Aggarwal} & \multicolumn{1}{c|}{Meeting Chairperson} \\ \hline
    Saif Fadhel                             & Meeting Facilitator                      \\ \hline
    Vanshaj Verma                           & Meeting Minute-Taker                     \\ \hline
    \end{tabular}
    \caption{Meeting Roles}
\end{table}

\section{Team Communication Plan}
\noindent The team has decided to use a Discord server for all communication related to this project. This allows for creation of multiple channels within the server to discuss different tasks that need to be completed. If needed, the group will be using Microsoft Teams and Emails to communicate to the TAs and the professor. The group will also use git issues to track and assign any issues within the project. 
\newpage 
\section{Team Member Roles}
\noindent The following table lists the roles of the members of the group:

\begin{table}[hbt!]
    \centering
    \begin{tabular}{|c|c|c|}
    \hline
    Name              & Role                                                                                           & Expertise                                                            \\ \hline
    Himanshu Aggarwal & \begin{tabular}[c]{@{}c@{}}Software Developer\\ Meeting Chairperson\end{tabular}               & \begin{tabular}[c]{@{}c@{}}Gantt\\ JavaScript\\ ReactJS\end{tabular} \\ \hline
    Saif Fadhel       & \begin{tabular}[c]{@{}c@{}}Software Developer\\ Meeting Facilitator\\ Team Leader\end{tabular} & \begin{tabular}[c]{@{}c@{}}LaTeX \\ JavaScript\end{tabular}          \\ \hline
    Vanshaj Verma     & \begin{tabular}[c]{@{}c@{}}Software Developer\\ Meeting Minute-Taker / Scribe\end{tabular}     & \begin{tabular}[c]{@{}c@{}}Git\\ JavaScript\end{tabular}             \\ \hline
    \end{tabular}
    \caption{Team Member Roles}
\end{table}

\section{Git Workflow Plan}
\noindent The team will use GitLab as a version control platform, and each individual team member will commit their initial changes to separate branches. The code will be tested and verified before merging into the \texttt{main} development branch to be approved by a merge request. Upon completion of the assigned deliverable, the final commit will be tagged in order to specify the version of the deliverable to be marked by TAs.

\section{Proof of Concept Demonstration Plan}
\noindent Implementing the data visualization charts for the historical time series data might be a challenge but we plan on researching how to properly implement them using \texttt{Echarts}. Testing will not be difficult because \texttt{Jest}, \texttt{React Testing Library} and \texttt{Cypress} provide easy ways to create and run tests. The required libraries can be easily installed via a Node Package Manager like \texttt{npm} and \texttt{yarn}. Portability is also not a concern as React supports all major browsers including Chrome, Safari, Firefox, etc.\\
\newline
\noindent We plan to demonstrate a working prototype of some of the tools we intend to implement. Our demonstration will include charts for data visualization and comparison, and tables for real-time aggregate data related to cryptocurrencies.


\section{Technology}
\noindent The project will use JavaScript as the primary programming language, along with HTML and CSS for the design. A React framework called \href{https://nextjs.org/}{NextJS} will be used as the frontend framework, along with JavaScript libraries like \href{https://react-query.tanstack.com/}{React Query} and \href{https://axios-http.com/docs/intro}{Axios} for fetching data and managing state. Libraries like \href{https://echarts.apache.org/en/index.html}{Echarts} and \href{https://apexcharts.com/}{ApexCharts} will be used to create data visualizations. Popular text editors like \href{https://atom.io/}{Atom} and \href{https://code.visualstudio.com/}{Visual Studio Code} will be used to add and modify code. \href{https://jestjs.io/}{Jest} and \href{https://github.com/testing-library/react-testing-library}{React Testing Library} will be used for Unit Testing, and \href{https://www.cypress.io/}{Cypress} will be used for End-to-End (E2E) and Integration Testing. The project will use {\href{https://github.com/reactjs/react-docgen}{reactjs/react-docgen}} for document generation.

\section{Coding Style}
\noindent The project will adhere to \href{https://developer.mozilla.org/en-US/docs/Web/JavaScript/Guide}{Mozilla's JavaScript Guide} as the coding standard to ensure consistency.

\section{Project Schedule}
\noindent The project schedule can be found \href{https://gitlab.cas.mcmaster.ca/webapp/webapp_l02_grp15/-/tree/main/ProjectSchedule}{here}.

\section{Project Review}
\textcolor{red}{The project was an intriguing preparation for Capstone projects in the future. It was a good learning experience working with a team, and dividing tasks amongst each team member who brought their own helpful specializations to the project. Each team member brought something new to the table and never missed a meeting, or group updates. No arguments have ensued over creative differences, and the project development went smoothly. Through developing the project, the team gained a lot more insight into Cryptocurrency as a whole, alongside gaining technological expertise in NextJS. The only criticism the team had regarding the content of the course was the use of the GanttProject software for the creation of Gantt Charts. The team was able to effectively book-keep meetings and divide the work using other software easier. \\ \\ Overall, this project was fun and exciting. We plan to continue working on this open-source project after the course and add features frequently.}

\end{document}


