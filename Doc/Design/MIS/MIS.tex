\documentclass[12pt]{article}

\usepackage{graphicx}
\usepackage{paralist}
\usepackage{hyperref}
\usepackage{xspace}
\usepackage{amsfonts}
\usepackage{amsmath}
\hypersetup{
    colorlinks,
    citecolor=black,
    filecolor=black,
    linkcolor=blue,
    urlcolor=blue
}

\newcommand{\latex}{\LaTeX\xspace}

\oddsidemargin 0mm
\evensidemargin 0mm
\textwidth 160mm
\textheight 200mm
\renewcommand\baselinestretch{1.0}

\pagestyle {plain}
\pagenumbering{arabic}

\newcounter{stepnum}

\title{SE 3XA3: Module Interface Specification\\CryptoMetrics}

\author{Team 15
		\\ Saif Fadhel, fadhels
		\\ Vanshaj Verma, vermav2
		\\ Himanshu Aggarwal, aggarwah
}

\date{March 18, 2022}

\begin {document}

\maketitle


\newpage
\tableofcontents

\newpage 
\section{Button Module}

\subsection{Module}
Button

\subsection{Uses}
None

\subsection{Semantics}

\subsubsection{Environment Variables}
\begin{tabular}{| l | l | p{8cm} |}
    \hline
    \textbf{Name} & \textbf{Type} & \textbf{Description}\\ \hline
    children & node & The content inside the button\\ \hline
    className & string & Additional CSS classes for the button\\ \hline
    onClick & function & The function to run when button is pressed\\ \hline
    type & string & The type of the button\\ \hline
    disabled & boolean & Whether the button is disabled\\ \hline
\end{tabular}

\subsubsection{State Variables}
None

\subsubsection{State Invariant}

None

\subsubsection{Assumptions}

None

\newpage

\section{ToggleButton Module}

\subsection{Module}
ToggleButton

\subsection{Uses}
Button

\subsection{Semantics}


\subsubsection{Environment Variables}
\begin{tabular}{| l | l | p{8cm} |}
    \hline
    \textbf{Name} & \textbf{Type} & \textbf{Description}\\ \hline
    children & node & The content inside the button\\ \hline
    setActive & function & The function to set the button active\\ \hline
    active & string & The id of the function that is currently active\\ \hline
    id & string & Unique identifier for the button\\ \hline
    className & string & Additional CSS classes for the button\\ \hline
\end{tabular}

\subsubsection{State Variables}
None

\subsubsection{State Invariant}
None

\subsubsection{Assumptions}
None

\newpage

\section{FilterButton Module}

\subsection{Module}
FilterButton

\subsection{Uses}
Button

\subsection{Semantics}
\subsubsection{Environment Variables}
\begin{tabular}{| l | l | p{10cm} |}
    \hline
    \textbf{Name} & \textbf{Type} & \textbf{Description}\\ \hline
    icon & node & The icon to be displayed\\ \hline
    onClick & function & The function to run when this button is pressed\\ \hline
\end{tabular}

\subsubsection{State Variables}
None

\subsubsection{State Invariant}
None

\subsubsection{Assumptions}
None

\newpage

\section{CryptoChartCard Module}

\subsection{Module}
CryptoChartCard

\subsection{Uses}
PlaceholderSkeleton \\
Queries 

\subsection{Semantics}
\subsubsection{Environment Variables}
\begin{tabular}{| l | l | p{10cm} |}
    \hline
    \textbf{Name} & \textbf{Type} & \textbf{Description}\\ \hline
    currencyName & string & The name of the cryptocurrency\\ \hline
    currencyId & string & The id of the cryptocurrency\\ \hline
    symbol & string & The symbol of the cryptocurrency\\ \hline
    icon & string & The icon of the cryptocurrency\\ \hline
    info & string & Any specific info for the cryptocurrency\\ \hline
    detail & string & Any specific detail for the cryptocurrency\\ \hline
    detailColor & string & A color for the detail text\\ \hline
    options & object & The options for the embedded chart\\ \hline
    type & string & The type of the chart\\ \hline
\end{tabular}

\subsubsection{State Variables}
None

\subsubsection{State Invariant}
None

\subsubsection{Assumptions}
None

\newpage


\section{CompareChart Module}

\subsection{Module}
CompareChart

\subsection{Uses}
ToggleButton\\
Dropdown \\
Queries

\subsection{Semantics}

\subsubsection{Environment Variables}
None

\subsubsection{State Variables}
\begin{tabular}{| l | l | p{10cm} |}
    \hline
    \textbf{Name} & \textbf{Type} & \textbf{Description}\\ \hline
    timerange & string & The current timerange for the chart\\ \hline
    firstCrypto & string & The id of the first cryptocurrency to compare\\ \hline
    secondCrypto & string & The id of the second cryptocurrency to compare\\ \hline
\end{tabular}

\subsubsection{State Invariant}
$firstCrypto \neq null \land
secondCrypto \neq null \land
timerange \neq null$

\subsubsection{Assumptions}
None

\newpage

\section{CryptoRowLineChart Module}

\subsection{Module}
CryptoRowLineChart

\subsection{Uses}
Queries

\subsection{Semantics}

\subsubsection{Environment Variables}
\begin{tabular}{| l | l | p{10cm} |}
    \hline
    \textbf{Name} & \textbf{Type} & \textbf{Description}\\ \hline
    currencyId & string & The id of the cryptocurrency\\ \hline
    color & string & The color of the chart\\ \hline
\end{tabular}

\subsubsection{State Variables}
None

\subsubsection{State Invariant}
None

\subsubsection{Assumptions}
None

\newpage

\section{Container Module}

\subsection{Module}
Container

\subsection{Uses}
None

\subsection{Semantics}

\subsubsection{Environment Variables}
\begin{tabular}{| l | l | p{10cm} |}
    \hline
    \textbf{Name} & \textbf{Type} & \textbf{Description}\\ \hline
    children & node & The content inside the container\\ \hline
\end{tabular}

\subsubsection{State Variables}
None

\subsubsection{State Invariant}
None

\subsubsection{Assumptions}
None

\newpage


\section{Main Module}

\subsection{Module}
Main

\subsection{Uses}
None

\subsection{Semantics}

\subsubsection{Environment Variables}
\begin{tabular}{| l | l | p{10cm} |}
    \hline
    \textbf{Name} & \textbf{Type} & \textbf{Description}\\ \hline
    children & node & The content inside the container\\ \hline
\end{tabular}

\subsubsection{State Variables}
None

\subsubsection{State Invariant}
None

\subsubsection{Assumptions}
None

\newpage

\section{Wrapper Module}

\subsection{Module}
Wrapper

\subsection{Uses}
None

\subsection{Semantics}

\subsubsection{Environment Variables}
\begin{tabular}{| l | l | p{10cm} |}
    \hline
    \textbf{Name} & \textbf{Type} & \textbf{Description}\\ \hline
    children & node & The content inside the container\\ \hline
\end{tabular}

\subsubsection{State Variables}
None

\subsubsection{State Invariant}
None

\subsubsection{Assumptions}
None

\newpage

\section{Dropdown Module}

\subsection{Module}
Dropdown

\subsection{Uses}
DropdownItem \\
Hooks

\subsection{Semantics}

\subsubsection{Environment Variables}
\begin{tabular}{| l | l | p{10cm} |}
    \hline
    \textbf{Name} & \textbf{Type} & \textbf{Description}\\ \hline
    list & array of objects & The list of options in the dropdown\\ \hline
    value & string & The current selected option in the dropdown\\ \hline
    setValue & function & The function to call when an option is selected\\ \hline
    disabled & array of strings & The id(s) of objects to disable in the dropdown\\ \hline
    className & string & Additional CSS classes for the button\\ \hline
\end{tabular}

\subsubsection{State Variables}
\begin{tabular}{| l | l | p{10cm} |}
    \hline
    \textbf{Name} & \textbf{Type} & \textbf{Description}\\ \hline
    open & boolean & Whether the dropdown component is open or closed\\ \hline
\end{tabular}

\subsubsection{State Invariant}
$open = true \lor open = false$

\subsubsection{Assumptions}
None

\newpage


\section{DropdownItem Module}

\subsection{Module}
DropdownItem

\subsection{Uses}
None

\subsection{Semantics}

\subsubsection{Environment Variables}
\begin{tabular}{| l | l | p{10cm} |}
    \hline
    \textbf{Name} & \textbf{Type} & \textbf{Description}\\ \hline
    children & node & The content inside the DropdownItem\\ \hline
    onClick & function & The function that is called when clicked on DropdownItem\\ \hline
    selected & boolean & A boolean that represents if the DropdownItem is selected\\ \hline
    disabled & boolean & A boolean that represents if the DropdownItem is disabled\\ \hline
\end{tabular}

\subsubsection{State Variables}
None
\subsubsection{State Invariant}
None

\subsubsection{Assumptions}
None

\newpage


\section{FilterDropdown Module}

\subsection{Module}
FilterDropdown

\subsection{Uses}
SecondaryFilterDropdown

\subsection{Semantics}

\subsubsection{Environment Variables}
\begin{tabular}{| l | l | p{10cm} |}
    \hline
    \textbf{Name} & \textbf{Type} & \textbf{Description}\\ \hline
    filterOptions & object & The options to display in the dropdown\\ \hline
    setOpen & function & The function to open/close the dropdown\\ \hline
    addFilter & function & The function to add the filter\\ \hline
    dropdownRef & node & Reference to the dropdown\\ \hline
\end{tabular}

\subsubsection{State Variables}
\begin{tabular}{| l | l | p{10cm} |}
    \hline
    \textbf{Name} & \textbf{Type} & \textbf{Description}\\ \hline
    selectedFilter & string & The id of the selected filter\\ \hline
    radioValue & string & The value of the selected radio field\\ \hline
    inputValue & string & The value of the input field\\ \hline
\end{tabular}

\subsubsection{State Invariant}
None

\subsubsection{Assumptions}
None

\newpage


\section{SecondaryFilterDropdown Module}

\subsection{Module}
SecondaryFilterDropdown

\subsection{Uses}
Button Module\\
RadioInputForm Module

\subsection{Semantics}

\subsubsection{Environment Variables}
\begin{tabular}{| l | l | p{10cm} |}
    \hline
    \textbf{Name} & \textbf{Type} & \textbf{Description}\\ \hline
    open & boolean & Whether the dropdown is open or closed\\ \hline
    onSelectedFilterChange & function & The function to call when a filter is modified\\ \hline
    radioOptions & object & The options for the radio fields\\ \hline
    radioValue & string & The value of the radio field\\ \hline
    onRadioChange & function & The function to call when radio value changes\\ \hline
    inputValue & string & The value of the input field\\ \hline
    onInputChange & function & The function to call when input value changes\\ \hline
    onFilterAdd & function & The function to call when a filter is added\\ \hline
    inputLeftSymbol & string & The symbol to display on the left side of the input field\\ \hline
    inputRightSymbol & string & The symbol to display on the right side of the input field\\ \hline
    inputType & string & The type of the input field\\ \hline

\end{tabular}

\subsubsection{State Variables}
None

\subsubsection{State Invariant}
None

\subsubsection{Assumptions}
None

\newpage

\section{Filter Module}

\subsection{Module}
Filter

\subsection{Uses}
Button

\subsection{Semantics}

\subsubsection{Environment Variables}
\begin{tabular}{| l | l | p{10cm} |}
    \hline
    \textbf{Name} & \textbf{Type} & \textbf{Description}\\ \hline
    subject & string & The subject of the filter\\ \hline
    condition & string & The condition of the filter\\ \hline
    value & string & The value of the filter\\ \hline
    symbolLeft & string & Any symbol on the left of the value\\ \hline
    symbolRight & string & Any symbol on the right of the value\\ \hline
    buttonIcon & node & The icon for the button\\ \hline
    onButtonClick & string & The function to call when button is clicked\\ \hline
\end{tabular}

\subsubsection{State Variables}
None

\subsubsection{State Invariant}
None

\subsubsection{Assumptions}
None

\newpage


\section{Filters Module}

\subsection{Module}
Filters

\subsection{Uses}
None

\subsection{Semantics}

\subsubsection{Environment Variables}
\begin{tabular}{| l | l | p{10cm} |}
    \hline
    \textbf{Name} & \textbf{Type} & \textbf{Description}\\ \hline
    children & node & The content inside the component\\ \hline
\end{tabular}

\subsubsection{State Variables}
None

\subsubsection{State Invariant}
None

\subsubsection{Assumptions}
None

\newpage

\section{Input Module}

\subsection{Module}
Input

\subsection{Uses}
None

\subsection{Semantics}

\subsubsection{Environment Variables}
\begin{tabular}{| l | l | p{10cm} |}
    \hline
    \textbf{Name} & \textbf{Type} & \textbf{Description}\\ \hline
    className & string & Additional CSS classes for the input\\ \hline
    placeholder & string & The placeholder for the input field\\ \hline
    type & string & The type of the input field\\ \hline
    onChange & function & The function to call when input value changes\\ \hline
    initialValue & string & The initialValue of the input field\\ \hline
    symbolLeft & node & The icon to display on the left side of the input field\\ \hline
    symbolRight & node & The icon to display on the right side of the input field\\ \hline
\end{tabular}

\subsubsection{State Variables}
None

\subsubsection{State Invariant}
None

\subsubsection{Assumptions}
None

\newpage

\section{SearchInput Module}

\subsection{Module}
SearchInput

\subsection{Uses}
Input

\subsection{Semantics}

\subsubsection{Environment Variables}
\begin{tabular}{| l | l | p{10cm} |}
    \hline
    \textbf{Name} & \textbf{Type} & \textbf{Description}\\ \hline
    onChange & function & The function to call when input value changes\\ \hline
\end{tabular}

\subsubsection{State Variables}
None

\subsubsection{State Invariant}
None

\subsubsection{Assumptions}
None

\newpage


\section{RadioInputForm Module}

\subsection{Module}
RadioInputForm

\subsection{Uses}
Radio \\
Input \\

\subsection{Semantics}

\subsubsection{Environment Variables}
\begin{tabular}{| l | l | p{10cm} |}
    \hline
    \textbf{Name} & \textbf{Type} & \textbf{Description}\\ \hline
    inputLeftSymbol & string & The symbol to display on the left side of the input field\\ \hline
    inputRightSymbol & string & The symbol to display on the right side of the input field\\ \hline
    inputType & string & The type of the input field\\ \hline
    options & object & The options for the radio fields\\ \hline
    radioValue & string & The value of the radio field\\ \hline
    onRadioChange & function & The function to call when radio value changes\\ \hline
    inputValue & string & The value of the input field\\ \hline
    onInputChange & function & The function to call when input value changes\\ \hline
    onSubmit & function & The function to call when the form is submitted\\ \hline
\end{tabular}

\subsubsection{State Variables}
None

\subsubsection{State Invariant}
None

\subsubsection{Assumptions}
None

\newpage


\section{Radio Module}

\subsection{Module}
Radio

\subsection{Uses}
None

\subsection{Semantics}

\subsubsection{Environment Variables}
\begin{tabular}{| l | l | p{10cm} |}
    \hline
    \textbf{Name} & \textbf{Type} & \textbf{Description}\\ \hline
    selected & string & It represents the value of the selected radio\\ \hline
    radioValue & string & It represents the value of the radio\\ \hline
    radioLabel & string & It represents the label of the radio\\ \hline
    onChange & function & It is run everytime the radio value changes\\ \hline
\end{tabular}

\subsubsection{State Variables}
None

\subsubsection{State Invariant}
None

\subsubsection{Assumptions}
None

\newpage


\section{PlaceholderSkeleton Module}

\subsection{Module}
PlaceholderSkeleton

\subsection{Uses}
None

\subsection{Semantics}

\subsubsection{Environment Variables}
\begin{tabular}{| l | l | p{10cm} |}
    \hline
    \textbf{Name} & \textbf{Type} & \textbf{Description}\\ \hline
    className & string & Additional CSS classes for the skeleton\\ \hline
\end{tabular}

\subsubsection{State Variables}
None

\subsubsection{State Invariant}
None

\subsubsection{Assumptions}
None

\newpage


\section{Sidebar Module}

\subsection{Module}
Sidebar

\subsection{Uses}
SidebarItem

\subsection{Semantics}

\subsubsection{Environment Variables}
\begin{tabular}{| l | l | p{10cm} |}
    \hline
    \textbf{Name} & \textbf{Type} & \textbf{Description}\\ \hline
    active & string & The id of the SidebarItem that is currently active\\ \hline
\end{tabular}

\subsubsection{State Variables}
None

\subsubsection{State Invariant}
None

\subsubsection{Assumptions}
None

\newpage


\section{SidebarItem Module}

\subsection{Module}
SidebarItem

\subsection{Uses}
None

\subsection{Semantics}

\subsubsection{Environment Variables}
\begin{tabular}{| l | l | p{10cm} |}
    \hline
    \textbf{Name} & \textbf{Type} & \textbf{Description}\\ \hline
    title & string & The title of the SidebarItem\\ \hline
    icon & node & The icon of the SidebarItem\\ \hline
    active & boolean & It represents if the SidebarItem is active\\ \hline
    to & string & It represents the link to the respective SidebarItem page\\ \hline
\end{tabular}

\subsubsection{State Variables}
None

\subsubsection{State Invariant}
None

\subsubsection{Assumptions}
None

\newpage


\section{Table Module}

\subsection{Module}
Table

\subsection{Uses}
Button

\subsection{Semantics}

\subsubsection{Environment Variables}
\begin{tabular}{| l | l | p{10cm} |}
    \hline
    \textbf{Name} & \textbf{Type} & \textbf{Description}\\ \hline
    children & node & The content inside the table\\ \hline
    className & string & Additional CSS classes for the table\\ \hline
\end{tabular}

\subsubsection{State Variables}
None

\subsubsection{State Invariant}
None

\subsubsection{Assumptions}
None

\newpage

\section{TableHeader Module}

\subsection{Module}
TableHeader

\subsection{Uses}
None

\subsection{Semantics}

\subsubsection{Environment Variables}
\begin{tabular}{| l | l | p{10cm} |}
    \hline
    \textbf{Name} & \textbf{Type} & \textbf{Description}\\ \hline
    children & node & The content inside the table header\\ \hline
    className & string & Additional CSS classes for the table header\\ \hline
\end{tabular}

\subsubsection{State Variables}
None

\subsubsection{State Invariant}
None

\subsubsection{Assumptions}
None

\newpage


\section{TableRow Module}

\subsection{Module}
TableRow

\subsection{Uses}
None

\subsection{Semantics}

\subsubsection{Environment Variables}
\begin{tabular}{| l | l | p{10cm} |}
    \hline
    \textbf{Name} & \textbf{Type} & \textbf{Description}\\ \hline
    children & node & The content inside the table row\\ \hline
    className & string & Additional CSS classes for the table row\\ \hline
\end{tabular}

\subsubsection{State Variables}
None

\subsubsection{State Invariant}
None

\subsubsection{Assumptions}
None

\newpage


\section{TableCell Module}

\subsection{Module}
TableCell

\subsection{Uses}
None

\subsection{Semantics}

\subsubsection{Environment Variables}
\begin{tabular}{| l | l | p{10cm} |}
    \hline
    \textbf{Name} & \textbf{Type} & \textbf{Description}\\ \hline
    children & node & The content inside the table cell\\ \hline
    className & string & Additional CSS classes for the table cell\\ \hline
\end{tabular}

\subsubsection{State Variables}
None

\subsubsection{State Invariant}
None

\subsubsection{Assumptions}
None

\newpage


\section{Tab Module}

\subsection{Module}
Tab

\subsection{Uses}
None

\subsection{Semantics}

\subsubsection{Environment Variables}
\begin{tabular}{| l | l | p{10cm} |}
    \hline
    \textbf{Name} & \textbf{Type} & \textbf{Description}\\ \hline
    content & node & The content to display\\ \hline
    onClick & function & The function to run when tab is clicked\\ \hline
    activeTab & string & The id of the currently active tab\\ \hline
    id & string & Unique identifier for the tab\\ \hline
\end{tabular}

\subsubsection{State Variables}
None

\subsubsection{State Invariant}
None

\subsubsection{Assumptions}
None

\newpage


\section{Tabs Module}

\subsection{Module}
Tabs

\subsection{Uses}
Tab

\subsection{Semantics}

\subsubsection{Environment Variables}
\begin{tabular}{| l | l | p{10cm} |}
    \hline
    \textbf{Name} & \textbf{Type} & \textbf{Description}\\ \hline
    children & array of nodes & The Tab elements to group together\\ \hline
\end{tabular}

\subsubsection{State Variables}
\begin{tabular}{| l | l | p{10cm} |}
    \hline
    \textbf{Name} & \textbf{Type} & \textbf{Description}\\ \hline
    tab & string & The id of the currently active tab\\ \hline
\end{tabular}

\subsubsection{State Invariant}
$tab \neq null$

\subsubsection{Assumptions}
None

\newpage



\section{BoldGradientHeading Module}

\subsection{Module}
BoldGradientHeading

\subsection{Uses}
None

\subsection{Semantics}

\subsubsection{Environment Variables}
\begin{tabular}{| l | l | p{10cm} |}
    \hline
    \textbf{Name} & \textbf{Type} & \textbf{Description}\\ \hline
    children & string & The text displayed inside the header\\ \hline
\end{tabular}

\subsubsection{State Variables}
None

\subsubsection{State Invariant}
None

\subsubsection{Assumptions}
None

\newpage


\section{Home Module}

\subsection{Module}
Home 

\subsection{Uses}
Container\\
Main\\
SearchInput\\
BoldGradientHeading\\
Queries\\
CryptoChartCard\\
Wrapper\\
Sidebar\\
Tabs\\
Tab\\
Hooks\\
FilterDropdown\\
Filter\\
Filters\\
FilterButton\\
CryptoRowLineChart\\
Table\\
TableCell\\
TableHeader\\
TableRow

\subsection{Semantics}

\subsubsection{Environment Variables}

\subsubsection{State Variables}
\begin{tabular}{| l | l | p{10cm} |}
    \hline
    \textbf{Name} & \textbf{Type} & \textbf{Description}\\ \hline
    dropdownOpen & boolean & Whether the dropdown is open\\ \hline
    searchText & string & The string to store the text being searched\\ \hline
    filters & object & The object for storing the applied filters \\ \hline
\end{tabular}

\subsubsection{State Invariant}
None

\subsubsection{Assumptions}
None

\newpage

\section{Hooks Module}

\subsection{Module}
Hooks

\subsection{Uses}
None

\subsection{Syntax}

\subsubsection{Exported Constants}

None

\subsubsection{Exported Types}

None

\subsubsection{Exported Access Programs}

\begin{tabular}{| l | l | l | p{5cm} |}
\hline
\textbf{Routine name} & \textbf{In} & \textbf{Out} & \textbf{Exceptions}\\
\hline
useFilters & Seq of objects & & \\
\hline
useOnClickOutside & node, function & object, function, function & \\
\hline

\end{tabular}

\subsection{Semantics}

\subsubsection{State Variables}

None

\subsubsection{State Invariant}

None

\subsubsection{Assumptions}

None

\subsubsection{Access Routine Semantics}


\noindent useFilters(Seq of objects):
\begin{itemize}
\item output: An object representing the applied filters, and two methods to add and remove a filter
\item exception: none
\end{itemize}

\noindent useOnClickOutside(node ref, function handler):
\begin{itemize}
\item output: none
\item exception: none
\end{itemize}



\newpage

\section{Queries Module}

\subsection{Module}
Queries

\subsection{Uses}
None

\subsection{Syntax}

\subsubsection{Exported Constants}

None

\subsubsection{Exported Types}

None

\subsubsection{Exported Access Programs}

\begin{tabular}{| l | l | l | p{5cm} |}
\hline
\textbf{Routine name} & \textbf{In} & \textbf{Out} & \textbf{Exceptions}\\
\hline
useCryptoTimeSeriesData & String, integer, String & Object & \\
\hline
useCryptoTimeSeriesRangeData & String, String, String & Object & \\
\hline
useCryptoList & String, integer, boolean & Object & \\
\hline

\end{tabular}

\subsection{Semantics}

\subsubsection{State Variables}

None

\subsubsection{State Invariant}

None

\subsubsection{Assumptions}

None

\subsubsection{Access Routine Semantics}


\noindent useCryptoTimeSeriesData(String name, integer days, String interval):
\begin{itemize}
\item output: An object consisting of the fetched API data
\item exception: none
\end{itemize}

\noindent useCryptoTimeSeriesRangeData(String name, String from, String to):
\begin{itemize}
\item output: An object consisting of the fetched API data
\item exception: none
\end{itemize}

\noindent useCryptoList(String currency, integer numberOfCurrencies, boolean sparkline):
\begin{itemize}
\item output: An object consisting of the fetched API data
\item exception: none
\end{itemize}

\newpage

\section{Major Revision History}
March 14, 2022 - Rough draft created of the sections \\
March 18, 2022 - Final draft of all of the sections complete \\


\end {document}

